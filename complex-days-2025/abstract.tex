\documentclass{article}
\usepackage{graphicx} % Required for inserting images
\usepackage{amssymb}

\title{Abstract for Complex days $2025$}
\author{Riccardo Daluiso}

\begin{document}

\maketitle

\section*{}

In the $n$-body problem, $n$ point masses move in space under their mutual gravitational interactions as described by Newton's theory of gravity. The forces acting between particles approach infinity when the mutual distances approach zero. Therefore, at collision, the equations of motion have singularities. Since Levi-Civita $(1903)$ and Sundman $(1907)$, the double collision has been "regularized", \emph{i.e.} the singularity has been made to disappear by means of algebraic transformations. In $1916$ Levi-Civita presented a regularization of the three-body problem which was ignored by almost all the authors with only a few exceptions. The idea of Levi-Civita was to first regularize a parabolic collision orbit of the Kepler problem by means of the Hamilton-Jacobi method, following procedures from Poincaré. These techniques surprisingly provide contact transformations which make the Hamiltonian of the three-body problem perfectly regular. 
More than fifty years later, Moser $(1970)$ compared the stereographic projection of the great circles of the sphere with Keplerian odograph, following ideas from Fock. This construction gives rise to a compactification of the phase space of the Keplerian orbits for a fixed level of energy together with the same regularization found by Levi-Civita, which is thus designated to be unique and fundamental. 
%
The main issue arising from these technique is the inability for the authors in finding regularizazions which work simultaneously for all the energy surfaces. Indeed, it turns out that the Levi-Civita transformation, defined by
\[
q \mapsto |p|^2q - 2 \langle  p , q \rangle p, \qquad p \mapsto \frac{p}{|p|^2},
\]
 if applied to the two-body problem sends all the collision states to the same value of the energy (\emph{i.e.} with the same value of the semi-major axis of the collision orbit) on the same state of the regularized system: the transformation obtained is not bijective. Albouy $(2024)$ proposes a simple trick to avoid this issue by introducing the energy transformed by the regularizing variables as a parameter. 

The purpose of this work is to apply the regularization methods to optimal control. Indeed, the compactification of the state space is relevant for the existence of solutions for optimal control problems in space mechanics, and we 
consider the control of a spacecraft under the attraction of one or more bodies. The main question we face is the extension of the theory to the case of non-constant energy.\\
 
 %When applied to the conservative Kepler problem, a regularization transforms the collision state in an harmonic oscillator, so the point is reflected along a trajectory with a velocity equal in modulus and opposite in sign to the one of collision. The control in the thrust of the particle could keep it going after the collision, but the gravity of the central body could make ejection slow, and thus the resulting trajectory could not be time-minimizing.

 \noindent Joint work with Alain Albouy (Observatoire de Paris) and Jean-Baptiste Caillau (UniCA).

\end{document}